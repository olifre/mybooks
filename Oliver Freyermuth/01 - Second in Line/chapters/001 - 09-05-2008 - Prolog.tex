\chapter{Prolog}
\label{cha:prolog}
Tief kniete er im Graben, um sich vor den Pfeilen des Feindes zu schützen. Sie würden kommen, es war nur eine Frage der Zeit, und sie mussten vorbereitet sein. Nur wenige Meter vor ihm hatte jemand sich vorgewagt und Schutz hinter einem viel zu flachen Stein gesucht. Es schien unmöglich, die Spannung zu ertragen, die sich in der heißen Wüstenluft ausbreitete, und auch er selbst war kurz davor, seine Deckung aufzugeben.

Doch ein einsamer Pfeil zischte heran, ein Vorbote dessen, was kommen musste. Nur Zentimeter neben ihm blieb seine Spitze im Sand stecken, und die roten Federn verrieten die Herkunft des gefürchteten Feindes. Ungläubig starrte er den Pfeil an und wartete, bis ein leichtes Knacken in der Ferne und das Rauschen heranschnellender Pfeile den Beginn des Kampfes besiegelten.
Nur einen Speer hatte er, und zehn Pfeile im Köcher. Schnell war der Bogen gespannt, und ohne sein Ziel zu sehen, schoss er im gleichen Winkel zurück, in dem die anderen Pfeile sein Versteck erreichten. 
Unwillig, nun mehr zu riskieren, wartete er ab. Und es kam, was geschehen musste.

Ein erstickter Schrei wenige Meter vor ihm verriet, dass es den Unvorsichtigen getroffen hatte. Nur wenige Worte hatten sie gewechselt, doch es waren genug gewesen, um diesen Menschen zu jemandem zu machen, der zu seinem Leben gehörte. 
Zwei Pfeile schoss er schnell hintereinander ab, signalisierte kurz nach links und rechts, dass sie ihm Feuerschutz geben sollten. Dann preschte er vor, kniete neben dem Mitkämpfer nieder und sah, aber realisierte nicht, dass dieser nur noch wenige Minuten zu Leben hatte. Schnell war er auf den Rücken gehievt und er schlich zurück, so schnell es möglich war. Dort, wo eben noch der Unvorsichtige gelegen hatte, sauste ein weiter Pfeil hernieder, doch es war zu spät. Mit Glück hatten sie den Graben erreicht, der Arm des Verletzten war abgebunden und der Pfeil entfernt. Nun hatte er eine reelle Chance, noch zu überleben. 
Doch seine Worte sagten etwas anderes.

\enquote{Wieso hast du mich gerettet? Ich wäre an vorderster Front gestorben! Aus einem Helden im Kampf hast du einen hilflosen\dots{}}
Etwas zerrte an mir, plötzlich war es kalt, ein Feuer knisterte nicht weit entfernt. Eine Hand lag auf meiner linken Schulter, erschrocken drehte ich den Kopf und merkte, dass ich die ganze Zeit in das Feuer gestarrt haben musste.

\enquote{Tariq, aufwachen! Wenn du unsere Gastfreundschaft genießen möchtest, musst du beim Essen auch wachbleiben.}
Die blonde Frau sah mich unverhohlen schelmisch an und reichte mir ein Tellergefäß aus Leder. Haferbrei war es, den sie mir anboten.

Immer noch etwas irritiert von der Wirkung eines einfachen Feuers begann ich schweigend, mich an dem Brei gütlich zu tun und dabei die Umsitzenden zu beobachten. Außer der Frau, die mich geweckt hatte, sah ich drei weitere Angehörige des Stammes der Apalachi sowie meinen langjährigen Freund und Kollegen Dr. Michael Krobeck, der die Operation leitete und unter dem Namen Toka unterwegs war. Auch er aß in Ruhe, sodass eine gewisse Spannung im Raum lag. Die Worte der Frau waren die einzigen gewesen, welche die Stille für wenige Momente unterbrochen hatten.

Sie war die Squaw des Häuptlings Akana, der mit seinen Söhnen Tokal und Tikono ihm gegenüber saß. 
Nur Krobeck wusste genau, warum sie hier waren. Es war klar, dass irgendetwas gefunden werden musste, denn so war es bisher immer gewesen.

Als das Essen beendet war, brach Akana schließlich die Stille. \enquote{Ihr müsst von weit her kommen. Was hat euch in das Land der Apalachi geführt?}

Krobeck war es, der antwortete: \enquote{Häuptling Akana, wir suchen etwas, das sich im Besitz eures Volkes befindet. Es ist das Wissen um einen Ort, der euch unter dem Namen \enquote{Ratana Ke'} bekannt ist. Wir wissen, dass euch dieser Ort heilig ist, und dass kein Fremder weiß, wo er sich befindet. Auch wir wollen ihn nur besuchen, ohne seinen genauen Ort kennenzulernen.}

\enquote{Ihr wisst, dass ihr damit etwas sehr Bedeutsames fordert? Diesen Ort hat niemals jemand gesehen, der nicht zum Stamm der Apalachi gehörte. Was lässt euch glauben, dass sich dies ändern wird?}

\enquote{Häuptling Akana, wir wissen sehr wohl, dass ihr keinen Grund habt, uns zu trauen. Wir sind Forschungsreisende, die nach Erkenntnis streben. Dennoch haben wir euch etwas anzubieten, was eure Meinung vielleicht ändern wird.}

Das war mein Stichwort gewesen. Ein Griff in meine rechte Tasche brachte ein Gerät zum Vorschein, das nicht von hier war. Unscheinbar, klein, aber wirkungsvoll war es, auch wenn es in der Lage war, so viel mehr zu tun als nur den Zweck zu erfüllen, für den es bestimmt war. 
Ich stand auf, verneigte mich vor dem Häuptling und überreichte ihm das handgroße Gerät. Schweigend nahm er es an und schaute fragend auf meinen Begleiter.

\enquote{Ihr müsst es auf einen eurer Feinde richten und den schwarzen Knopf drücken}, erklärte Krobeck.\enquote{Dieser wird dann für eine gewisse Zeit von unsichtbaren Fesseln gehalten.}

\enquote{Wie kommt ihr an solches Teufelswerk?}

\enquote{Wir selbst haben es erschaffen, Häuptling.}

\enquote{Dann müsst ihr Gesandte der Götter sein, die uns in der Gestalt Fremder besuchen. Wenn dem wirklich so ist, dürft ihr unsere heilige Stätte besuchen.}

Mit diesen Worten erhob sich Häuptling Akana und verließ das Zelt. Ich setzte meine Brille auf, und Krobeck tat es mir gleich. Es war nur ein kleiner Gedanke, und das Zelt verschwand --- da war Akana, er ging auf den Fluss nahe der Siedlung zu. Ein Wildschwein in einiger Entfernung auf der anderen Seite des Flusses war sein Ziel: Mit einem meisterhaften Schuss erreichte die Strahlenwaffe vollkommen geräuschlos ihr Ziel. Ein blauer Lichtblitz traf, verteilte sich über den fremden Körper und verschwand.

Für die nächsten drei Stunden würde das Tier in dieser Situation verharren. Nach einem zunächst ungläubigen Blick wartete Akana wenige Minuten ab --- schließlich kehrte er zurück und nahm seinen vorherigen Platz am Feuer wieder ein.

\enquote{Hat euch die Wirkung auf das Tier überzeugt?}, fragte Krobeck mit starrem Blick. Und tatsächlich mischte sich für einen kurzen Moment Entsetzen in das Antlitz des Häuptlings, doch er fasste sich sehr schnell wieder.

\enquote{Ihr müsst wirklich Gesandte der Götter sein}, stellte er fest und verneigte sich schließlich. \enquote{Meine beiden Brüder Tokal und Tikono werden euch gleich zu eurem Ziel führen. Es ist nicht weit entfernt, doch gut bewacht. Habt ihr weitere Wünsche? Mein Volk ist gern bereit, sie euch zu erfüllen.}

\enquote{Nein, das wäre wohl für's Erste genug}, antwortete Krobeck selbstzufrieden. Ich saß wie immer schweigend da und betrachtete die Situation. Langsam und mit gespieltem Stolz erhob sich Krobeck, und ich folgte ihm. Tokal und Tikono hoben bereits das Fell, welches das Zelt verschlossen hatte, sodass wir passieren konnten.

Kälte umfing mich, als ich den ersten Schritt aus dem Zelt tat. Kälte und eine seltsame Stille, die sich über das Dorf gelegt hatte. Vereinzelt sah man Gesichter, die aus den Zelten starrten. Alleine die Tatsache, dass die Brüder des Häuptlings selbst uns begleiteten, musste ihnen Respekt einflößen.

Es war tatsächlich kein weiter Marsch. Nur wenige Kilometer führten uns am Fluss entlang vom Dorf weg, über einen mit Büschen und Ästen versteckten Pfad und schließlich an eine Felswand. Hier musste es sein. Tokal ging voraus und Tikono bildete den Schluss der Kette. Ich konnte nicht umhin, für wenige Sekunden mit meiner Brille die Umgebung abzusuchen. Mir fiel erst jetzt auf, dass keiner der Einwohner sie nur mehr als eines kritischen Blickes gewürdigt hatte. Götter wurden nicht hinterfragt.

Nicht weit von uns hatten sich Indianer im Gebüsch versteckt, und als ich nach oben sah, konnte ich auch in den Bäumen Wachen entdecken. Der Gedanke, dass in dieser Welt bereits mit Terahertz-Strahlung gearbeitet wurde, erschien mir plötzlich wie ein schlechter Witz. Doch es war niemals anders gewesen.

Schließlich hatten wir den Felsvorsprung erreicht, der zu der Höhle führen musste, denn Tokal blieb stehen und von links und rechts kamen zwei Wachen aus dem Schatten der Felsen hervor. Ein Blick nach vorne verriet mir, dass hinter festem Fels eine Höhle liegen musste, und ich sah Krobeck überrascht an. Auch er sah erstaunt zurück. Wie sollten Indianer durch festen Fels jemals hineinkommen? Eine Öffnung war nicht zu sehen, zumindest keine, die mit Terahertz-Strahlung sichtbar würde. Ebenso wenig schienen kleine Risse im Fels erkennbar, normalerweise ein Hinweis auf einen eingearbeiteten Mechanismus.

Die Wachen traten zur Seite, nachdem sie Tokal und seinen Bruder begrüßt hatten. Sie erklärten uns, dass wir eine spezielle Stelle auf dem Felsvorsprung vor der Höhle betreten müssten. Dann würde einer der Wächter den Mechanismus aktivieren. Es war tatsächlich überraschend, wie leicht ein Indianerstamm Fremden Vertrauen entgegenbrachte, wenn diese als Gesandte der Götter identifiziert waren. Auch diesmal war ich es, der den ersten Schritt tun würde. Langsam lehnte ich mich gegen die Felswand und wartete. Harter, kalter Stein war hinter mir.

Und Sekunden später sah ich nur noch Dunkelheit. Ein weiterer Gedanke deaktivierte die Brille, oder sollte es zumindest tun, denn nichts geschah. Nur eine kleine rote Lampe leuchtete unten rechts --- leuchtete? Normalerweise blinkte die LED im Sekundentakt\dots{}

Der Zeitfluss stand still. Das war es sicher nicht, was geschehen sollte. Der Effekt hatte von allem außer mir selbst Besitz ergriffen, und als ich die Brille abnahm, konnte ich sie alle sehen. Plötzlich schweigend, still, ruhig. Krobeck mit seinem etwas forschen Gesichtsausdruck und den zusammengezogenen Augenbrauen, Tokal und Tikono ehrerbietig und gleichzeitig ehrwürdig mit festem, nicht jedoch starrem Blick.

Doch etwas war anders. Der kalte Fels hinter mir war verschwunden\dots{}
Hektisch drehte ich mich um, und sah doch nur den grauen Fels, der auch vorhin dagwesen war. Als ich jedoch die Hand ausstreckte und ihn berühren wollte, fasste ich einfach hindurch. Ein leichtes Prickeln war zu spüren. Es war ein moduliertes Kraftfeld, welches so hochfrequent sein musste, dass es --- bei normalem Zeitfluss --- sogar Terahertz-Strahlung reflektierte.

Eine Nachricht für Krobeck würde ich nicht hinterlassen können, dazu fehlte im wahrsten Sinne des Wortes die Zeit. 
Nun war es nur noch ein Schritt hinein, ein erneuter Schritt ins Ungewisse. 
