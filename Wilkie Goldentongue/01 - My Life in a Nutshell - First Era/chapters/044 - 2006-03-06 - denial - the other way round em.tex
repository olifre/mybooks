\chapter{Denial --- The other Way Round!}
\label{cha:denial-the-other-way-round}
\subsection*{Originally published: \DTMDate{2006-03-06}}
\begin{quote}
It's me, again!

Another text has just arrived, though time is rushing by in a terrible speed; just now, three hours have passed without me noticing it. And nothing had been achieved in that now forever lost time. But we won't go on discussing about that; we shall begin.

New facts: Denial --- The other Way Round!
\end{quote}

The weekend had been pretty informative, at least Sunday evening. Then, he'd learned something about a girl; it shall be pretty easy to find a name for her, as we could simply call her R., as she could be symbolized by the colour and the plant of a rose; however, not a red, but a rose coloured more pale, close to pink. But as this colour has been already assigned to somebody else, we should not make a fuss about it. On the other hand, we may probably not talk about that person at all in the next time, as she seems to try to evade our protagonist after some communication which seemed quite interesting took place. So, you may ask, what makes her more important than all the other people he knows? Several things: Some people --- among them his best friend --- are in love with that girl, and though she's currently not with anybody (that had been finished some month ago), all those people are too shy to approach her. About one year ago, she had been following his once-best-friend (the second one, who preceded the current one) nearly everywhere, and some had been assuming that something was \enquote{under construction}, so to say. That was finished when this boy found himself a girlfriend; Nevertheless, the friendship would still last.

He was reminded of all the things he'd experienced together --- or not together --- with O. That was gone, but she still assumed some kind of friendship, as they smiled at each other again this day; He'd seen her for such a long time that Monday --- at least, in comparison to the last weeks --- that he was stupefied when noticing that his feelings seemed to be really \emph{gone}. Of course, he knew that they were still hidden somewhere and could be resurrected easily --- but who'd do that job?

Real love was senseless, and just an invention of evolution to persevere itself.

If she didn't do it, nobody would; but there were more interesting things happening today. Apart from the fact, that P. seemed to tell him about everything \emph{(things a girl would normally only tell --- well, another girl, which was a good --- or the best --- friend of her's)}, she also didn't mind her leg touching his, or her hand touching his; it had become normal to her. However, when he was sitting somewhere with somebody else, and she \emph{could} safely walk up to him, she didn't. She'd rather stay with one of her girlfriends. Nevertheless, she'd told him about people he didn't know, but he'd got to know about from her descriptions; somebody who was after her, though he'd perfectly fit with somebody else, and P. simply couldn't make him realize that she wasn't interested. She kept explaining our protagonist things like these, and he wasn't sure what he'd make of them.

On the other hand, P.'s friend would do something similar; well, she didn't tell him anything about herself anymore, but she talked to him about this and that, about the work both had to do and the future that was to come. All those people around him seemed to be so open-minded, and he was just adjusting himself.

Slowly, he'd fit into the scheme that made this part of the world. 
But he didn't know if that really was the thing wanted to achieve. Well, he didn't have any choice, as this was his nature, and he couldn't act disregarding it. 
But the most interesting thing about that day was the way other people seemed to be judging this small group.

A small group --- he'd always liked to think about groups of friendship, and at the same time, he'd tried to take part in all of them. Thus, he was everybodies' friend; and even that problem of being fixed with one group had been solved, as he was now accepted by all those people as a fixed member. All those boys and girls accepting him as one of their own kind\dots{}

R. had once, that day, ran close to somebody behind him; it was B. she was heading to. However, he'd thought for some seconds that she was trying to talk to him, explaining the sudden silence --- she wasn't. 
The \enquote{Evil One with the Glasses} had been with him again that morning, and though both hadn't dared to look into each other's eyes after he hadn't reserved a seat for her, they'd talked after having left the bus. And he'd shocked her with his knowledge about all those tiny details out there --- and he'd once and for all noticed the wonder of that knowledge, which had till then not been approved by anybody. Nevertheless, she was merely shocked and seemed to take him for somebody with a crazy spleen.

He'd also had contact to Y. that day, whom he'd helped a bit before talking about less important things --- in fact, just smalltalk without any importance at all.

She was still the old kind of person she'd always been: Trusting everybody, and at the same time expressing her feelings openly. Like the sun, which was one of the reasons why she had been assigned the colour of yellow. He felt that she was the kind of person that would wait for the partner to advance, if a new relationship was to develop; probably, he would have had a chance, but he knew that she was too different, as she took some things too simply.

A cunning plan was yet about to be put into practice, and if there were some people who'd join into this group of learning together he wanted to set up, his methods of understanding could be passed on. He'd already contacted O.'s best friend, as you may remember --- but the silence he'd received in return was killing him. That morning, he hadn't greeted neither O. nor her best friend, but when her best friend had greeted him, they had all greeted each other out of a sudden. And when he'd wished to ask O. about something that day, he just stood there, pondering what to do. But O. was suddenly in a hurry, and he felt he should let her go; some minutes later, when she'd pass him by, he'd asked her something, and though she seemed to be in a hurry again, she thought for a moment, told him she didn't know and excused for that lack of knowledge.

This excuse took him by surprise, as it was something that was indeed \emph{not} normal for anybody. Well, she was different, we all know about that; but this thing was peculiar. He hadn't really thought about it till now, when he was thinking about it in all detail, realizing that it could just have been something she'd done by chance or an expression of the wish to go on talking\dots{}

He wouldn't know, and he wouldn't try to find out. Something else had happened that day, something of vital importance: 
P.'s friend had denied something, somebody: \textbf{him}. When somebody assumed those two being a nice couple, she'd said that he was too old.

But she'd said it pretty quietly, probably expecting him to react; but he was stupefied, and not prepared to any new social problems approaching his stable system. In fact, he felt hurt --- at least, a bit. On the other hand, he felt that this gave him some sort of security, knowing about her attitude; but some voice kept telling him she didn't mean it, and the truth was not to be presented so easily. The problem was that he was a kind of person who wouldn't invest too much power in finding out about it, as he knew it wouldn't really be beneficial for him.

That was the \enquote{Denial}; and for the first time in his life, it had been the other way round. Today would have been the birthday of his older brother, who'd died after having lived for several weeks. 
The difference of age between the two of them would have been pretty equal to the difference between him and P.'s friend; something around two years seperated --- or unified --- them. 
Everything is a matter of perspective.

We don't want to go into detail about this, as the only things he knows about his brother were his name, his story --- and his grave. 
The grave --- the end of every life, and the beginning of everything.

He'd encountered some peculiar conversations with B. that day; using puns and some interesting information to please her, he was finally able to talk to her for more than one hour, though he finally felt like a chatter box. He knew that she was just being friendly, and nothing more; but he couldn't help but trying to make friends with her, in the same way he wanted to be everybodies' friend. And she seemed to be interesting.

B.-B. hadn't been there that day. She had been absent, and he couldn't remember when he'd seen her the last time.

Another girl, which was also emanating the colour of black in some way, an interesting person, also seemed to like talking to him. When he'd walk up to her, without saying a word, she'd start telling him what she was doing and which problems she had, giving him the chance to solve them. On the other hand, the boy that was interested in her since --- let's say, a year --- had told her so publicly, and everybody would know about it. Till now, she had just accepted him as a friend, in some similar way she'd accepted our protagonists, while that one hadn't had to work that hard to win her friendship. 
So, how can we conclude that day full of action?

He felt dumped, as P.'s friend had denied him, and at the same time more insecure than at any time before. 
He felt accepted by everybody, but was still searching for \textbf{\enquote{the one and singular}}. Was she among the people he knew? He wouldn't deny her existence, as he didn't want to deny himself. 
He felt he didn't know a thing about other people's thoughts, and he knew he knew more than most people did. 
Left in this confusion, we'll finish this one for today.

More is yet to come, though you shouldn't expect things to become any clearer\dots{}
Opinions?

\begin{quote}
Denying \\
the facts \\
was the worst \\
the best \\
the singular thing to do. \\
It was the task of emotions \\
to deny the facts; \\
and it was the task of logic \\
to accept this denial. \\
--- W.G.
\end{quote}

\begin{quote}
Being bashful \\
is the source of loneliness \\
blissfulness, \\
success and breakthrough. \\
Bashfulness can be the key \\
to a stable relationship; \\
or to impulsive, short partnerships, \\
ending in loneliness. \\
--- W.G.
\end{quote}
